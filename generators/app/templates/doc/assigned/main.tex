%
% Module TODO:MODNUM Chapter TODO:CHAPNUM Program Documentation
% CSC160-C00: Computer Science I (C++)
% Author: Ashton Hellwig
%


\documentclass[a4paper,11pt]{article}


  % Packages
  \usepackage[english]{babel}       % Internationalization
  \usepackage{soul}                 % Highlighting
  \usepackage{hyperref}             % Links (internal and external)
  \usepackage{fancyhdr}             % Headers and footers
  \usepackage[dvipsnames]{xcolor}   % Text Colors
  \usepackage{listings}             % Code Snippets
  \usepackage{algorithmicx}         % Algorithmic notation support
	\usepackage{algpseudocode}        % Algorithmic notation environments
	\usepackage{enumitem}             % Ordered lists
  \usepackage{geometry}             % Page layout
  \usepackage{graphicx}             % Image support
  \usepackage[toc, page]{appendix}  % Appendix
  \usepackage{amsmath}              % Mathematical Typesetting


  % Colors
  \newcommand{\commentstylecolor}{\color{Gray}}
  \newcommand{\keywordstylecolor}{\color{MidnightBlue}}
  \newcommand{\stringstylecolor}{\color{ForestGreen}}
  \newcommand{\questioninput}{\color{Red}}
  \newcommand{\answertcolor}{\color{Green}}
  \newcommand{\myanswer}{\answertcolor{\hl}}


  % Image Directory
  \graphicspath{ {screenshots/} }


  % Hyperlink Setup
  \hypersetup{
    colorlinks = true,
    urlcolor = blue,
    linkcolor = blue
  }


  % Syntax-Highlighting for Code Snippets
  \lstset{
    backgroundcolor=\color{white},
    breaklines=true,%
    captionpos=b,%
    frame=tb,%
    tabsize=4,%
    numbers=left,%
    showstringspaces=false,%
    commentstyle=\commentstylecolor,%
    keywordstyle=\keywordstylecolor,%
    stringstyle=\stringstylecolor%
  }


  % Page Configuration
  %% Style
  \pagestyle{fancy}

  %% Layout
  \geometry{%
  a4paper,%
  top=2.5cm,%
  bottom=2.5cm,%
  left=2.5cm,%
  right=2.5cm%
  }
  \setlength{\headheight}{12pt}
  \setlength{\floatsep}{12pt}

  %% Title page
  \title{Chapter 5 Program Documentation}
  \author{Ashton Hellwig}
  \date\today
  \setcounter{tocdepth}{3}

  %% Subsequent pages
  \lhead{CSC160}
  \rhead{Computer Science I (C++)}
  \lfoot{MTODO:MODNUMCTODO:CHAPNUM}
  \rfoot{A. Hellwig}


  % Document Content
\begin{document}
  % Title Page
  \maketitle
  \tableofcontents
  \listoffigures
  \newpage


  % Problem Analysis
  \section{Problem Analysis}
    The problem states:
    \begin{quotation}
      Write a program that uses \textbf{while} loops to perform the following:
      \begin{enumerate}
        \item Prompt the user to input two integers: firstNum and secondNum
          (firstNum must be less than secondNum).
        \item Output all odd numbers between firstNum and secondNum.
        \item Output the sum of all even numbers between firstNum and secondNum.
        \item Output the numbers and their squares between 1 and 10.
        \item Output the sum of the square of the odd numbers between firstNum
          and secondNum.
        \item Output all uppercase letters.
      \end{enumerate}
    \end{quotation}

    \subsection{Data}
      Available data includes:
      \begin{enumerate}
        \item There are two variables: \texttt{firstNum} and \texttt{secondNum}
        \item \texttt{firstNum} must always be less than \texttt{secondNum}
      \end{enumerate}

    \subsection{Desired Output}
      \begin{figure}[h]
        \caption{main.cpp output}
        \begin{lstlisting}[language=bash]
Odd numbers between firstNum and secondNum:
Sum of even numbers between firstNum and secondNum:
firstNum =
firstNumSquares between 1 and 10: 
secondNum =
secondNumSquares between 1 and 10:
The sum of the square of the odd numbers between firstNum and secondNum =
All uppercase letters used were:
        \end{lstlisting}
        \label{fig:do}
      \end{figure}


  % Algorithm
  \newpage
  \section{Algorithm}
    Below is the algorithm for the program.
    \begin{figure}[h]
      \caption{Chapter 5 Program Algorithm}
      \vspace{12pt}
      \begin{algorithmic}
        %% Variables
        \State \Comment{--Variables--}
        \State 
        \State $firstNum\gets $
          \Comment{Needs user input}
        \State $secondNum\gets $
          \Comment{Needs user input}
        %% Prompt Lines
        \State \Comment{--Prompt Lines--}
        \State \Call{toOutput}{} ``Please enter the values for firstNum''
          \State $firstNum\gets input$
        \State \Call{toOutput}{} ``Please enter the values for secondNum''
          \State $secondNum\gets input$
      \end{algorithmic}
      \label{alg:c5}
    \end{figure}


  % User Documentation
  \newpage
  \section{User Documentation}

    %% Usage
    \subsection{Build}
      The following are instructions with two use cases:
      \begin{itemize}
        \item Within Visual Studio 2017
        \item Bundled Release
        \item with GNU G++
      \end{itemize}
      \subsubsection{Within Visual Studio}
        Simply load \texttt{ChapterTODO:CHAPNUM.sln} in Microsoft Visual Studio and 
          build/run the \texttt{release} version. If you require debugging
          information, switch the configuration to \texttt{debug}.
      \subsubsection{Bundled Release}
        \begin{enumerate}
          \item Navigate to the unziped folder containing the binary, 
            \textbf{with a terminal emulator or command prompt}, this will
            (most likely) mean running:
            \begin{lstlisting}[language=bash]
              cd %USERPROFILE%\Downloads\ChapterTODO:CHAPNUM\x64\Release\
            \end{lstlisting}
          \item To run the program simply issue this within the command
            prompt
            \begin{lstlisting}[language=bash]
              .\ChapterTODO:CHAPNUM.exe
            \end{lstlisting}
        \end{enumerate}
        Of course if preferred, you may also navigate to the release folder in
          file explorer and double click the executable (\texttt{Chapter4.exe})
      \subsection{With g++}
        If you prefer to use an open source debugger and compiler then I assume
          the following:
          \begin{enumerate}
            \item You have installed \href{http://www.mingw.org/}{MinGW} and
              it is in your \texttt{\$PATH}
            \item You have installed the
            \href{http://www.mingw.org/wiki/MSYS}{MSYS Tools} and they are in
              your \texttt{\$PATH}
          \end{enumerate}


  % Appendix
  \newpage
  \appendix

  % Appendix A
  \section{Appendix A}

\end{document}
